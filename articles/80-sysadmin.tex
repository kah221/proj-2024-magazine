\documentclass[a4paper,11pt]{jsarticle}


% 数式
\usepackage{amsmath,amsfonts}
\usepackage{bm}
% 画像
\usepackage[dvipdfmx]{graphicx}
% url
\usepackage{hyperref}

\begin{document}


\title{システム管理者のお仕事紹介}
\author{豊嶋択斗(とよしまたくと)}
\date{\today}
\maketitle

\section{はじめに}
島根大学ものづくり部Pim(以降Pim)では、現役部員やOBの方が交流するチャットツールとしてMattemostというslackライクなOSSを使用している。 PimではこのMattermostをセルフホストで運用しており、その管理を担っているのが\textgt{システム管理者}である。本記事ではキャラの濃いメンバーが集いしPimのシステム管理者の一人である私、豊嶋が知られざるシステム管理者のお仕事内容について簡単に紹介していきます。

\section{自己紹介}
本題に入る前に、まずは簡単に自己紹介をします。豊嶋といいます。2回生で、所属は知能情報デザイン学科(いわゆる情報科)です。好きな言葉は最安値、嫌いなものはセール価格を自称する通常価格な商品です。頭にローカルで動作する食料品カテゴリだけの価格.c〇mがインストールされている(と自称する)人です。どうぞよろしく。

\section{システム管理者のお仕事}
システム管理者のお仕事は大きく分けて以下の3つです
-\begin{enumerate}
  \item サーバの保守管理
  \begin{enumerate}
    \item Mattermost
    \item Redmine
  \end{enumerate}
  \item Website (\url{https://www.pim.gr.jp} など)
  \item メール
  \item PimSignage
\end{enumerate}

\section{サーバの保守管理}
サーバ機本体の管理とサーバ機で動かしているサービスの管理をここでは指します。サーバ機本体はUbuntuを入れて運用しています。定期的にパッケージの更新をすることや、サーバは大学に設置されておりますので、大学で数日に渡る計画停電が行われる間サービスを利用できるように、サーバを管理者宅へ一時的に移設することをしています。今年は我が家にサーバ君が2週間弱お泊りに来ていました。来た当初は「意外とうるさいなこいつ」と就寝時に少し気になっていたのですが、いなくなった後は部屋が静かになって少し寂しい気持ちになりました。

さて、話を戻しまして次にサーバ機で動かしているサービスについて説明します。オンプレミスで運用しているサービスはMattermost,Redmineの2つです。Zabbixという監視用OSSを用いて、サーバが予期せず落ちていないか等監視しており、障害が発生したらシステム管理者が対応に当たっています。Mattermostに関するお仕事内容は主に、月に1回入る大きな更新を行うことです。この大きな更新はほとんどの場合15,16日辺りにリリースされるので、Pimが毎月10日に行っている部会時に更新内容を紹介することもしています。
Redmineは弊部の運営メンバーがタスク管理、進捗共有を目的に活用しているOSSになります。Redmineも、Mattermost程高頻度ではありませんが、更新しています(私はしたことないですが)。

\section{Website}
Websiteでは弊部の紹介や活動予定を公開しています。私はしていないのですが、websiteそのものの作成や公開するまでの設定、websiteの編集を行っています。ちなみに、 \url{https://www.pim.gr.jp} はCloudflare Pagesを用いて公開されています。

\section{メール}
弊部はさくらインターネットが提供する「さくらのメールボックス」を利用しています。お仕事内容として主に、新規にメールアドレスが必要になった際に作成することやメールボックスの容量が一定値を超えそうなユーザに対して通知を行っています。

\section{PimSignage}
PimSignageは弊部が提供しているデジタルサイネージサービスのことです。島根大学生物資源学部3号館118室 数理データサイエンス教育研究センターにて、学内サークルの広告を掲載しています。システム管理者のお仕事は主に、問い合わせから掲示まのやり取り及び、掲示物の反映を行っております。デジタルサイネージの詳細は \url{https://www.pim.gr.jp/services/pim-signage} に記載されています。弊部では担当しておりませんが、学外の方向けのサービスもございますので良ければご覧ください。

\section{終わりに}
以上、簡単にはなりますがシステム管理者のお仕事紹介でした。

\end{document}
